

%----------------------------------------------------------------------------------------
%	PACKAGES AND OTHER DOCUMENT CONFIGURATIONS
%----------------------------------------------------------------------------------------

\documentclass{article}

\usepackage{fancyhdr} % Required for custom headers
\usepackage{lastpage} % Required to determine the last page for the footer
\usepackage{extramarks} % Required for headers and footers
\usepackage[usenames,dvipsnames]{color} % Required for custom colors
\usepackage{graphicx} % Required to insert images
\usepackage{listings} % Required for insertion of code
\usepackage{courier} % Required for the courier font
\usepackage[table]{xcolor}
\usepackage{multirow}
\usepackage{tabularx}

% Margins
\topmargin=-0.45in
\evensidemargin=0in
\oddsidemargin=0in
\textwidth=6.5in
\textheight=9.0in
\headsep=0.25in

\linespread{1.1} % Line spacing

% Set up the header and footer
\pagestyle{fancy}
\lhead{\hmwkAuthorName\ : \hmwkStudentID} % Top left header
\chead{ \hmwkClassShort} % Top center head
\rhead{\firstxmark } % Top right header
\lfoot{\lastxmark} % Bottom left footer
\cfoot{} % Bottom center footer
\rfoot{Page\ \thepage\ of\ \protect\pageref{LastPage}} % Bottom right footer
\renewcommand\headrulewidth{0.4pt} % Size of the header rule
\renewcommand\footrulewidth{0.4pt} % Size of the footer rule

\setlength\parindent{0pt} % Removes all indentation from paragraphs

%----------------------------------------------------------------------------------------
%	CODE INCLUSION CONFIGURATION
%----------------------------------------------------------------------------------------

\definecolor{MyDarkGreen}{rgb}{0.0,0.4,0.0} % This is the color used for comments
\lstloadlanguages{C} % Load Perl syntax for listings, for a list of other languages supported see: ftp://ftp.tex.ac.uk/tex-archive/macros/latex/contrib/listings/listings.pdf
\lstset{language=SQL, % Use c in this example
        frame=single, % Single frame around code
        basicstyle=\small\ttfamily, % Use small true type font
        keywordstyle=[1]\color{Blue}\bf, % cfunctions bold and blue
        keywordstyle=[2]\color{Purple}, % c function arguments purple
        keywordstyle=[3]\color{Blue}\underbar, % Custom functions underlined and blue
        identifierstyle=, % Nothing special about identifiers                                         
        commentstyle=\usefont{T1}{pcr}{m}{sl}\color{MyDarkGreen}\small, % Comments small dark green courier font
        stringstyle=\color{Purple}, % Strings are purple
        showstringspaces=false, % Don't put marks in string spaces
        tabsize=5, % 5 spaces per tab
        %
        % Put standard c functions not included in the default language here
        morekeywords={rand},
        %
        % Put c function parameters here
        morekeywords=[2]{on, off, interp},
        %
        % Put user defined functions here
        morekeywords=[3]{test},
       	%
        morecomment=[l][\color{Blue}]{...}, % Line continuation (...) like blue comment
        numbers=left, % Line numbers on left
        firstnumber=1, % Line numbers start with line 1
        numberstyle=\tiny\color{Blue}, % Line numbers are blue and small
        stepnumber=1 % Line numbers go in steps of 5
}

% Creates a new command to include a script, the first parameter is the filename of the script (without .txt), the second parameter is the caption
\newcommand{\Cscript}[2]{
\begin{itemize}
\item[]\lstinputlisting[caption=#2,label=#1]{#1.txt}
\end{itemize}
}
\renewcommand{\arraystretch}{1.5}

%----------------------------------------------------------------------------------------
%	DOCUMENT STRUCTURE COMMANDS
%	Skip this unless you know what you're doing
%----------------------------------------------------------------------------------------

% Header and footer for when a page split occurs within a problem environment
\newcommand{\enterProblemHeader}[1]{
\nobreak\extramarks{#1}{#1 continued on next page\ldots}\nobreak
\nobreak\extramarks{#1 }{#1 continued on next page\ldots}\nobreak
}

% Header and footer for when a page split occurs between problem environments
\newcommand{\exitProblemHeader}[1]{
\nobreak\extramarks{#1}{#1 continued on next page\ldots}\nobreak
\nobreak\extramarks{#1}{}\nobreak
}

\setcounter{secnumdepth}{0} % Removes default section numbers
\newcounter{homeworkProblemCounter} % Creates a counter to keep track of the number of problems

\newcommand{\homeworkProblemName}{}
\newenvironment{homeworkProblem}[1][Problem \arabic{homeworkProblemCounter}]{ % Makes a new environment called homeworkProblem which takes 1 argument (custom name) but the default is "Problem #"
\stepcounter{homeworkProblemCounter} % Increase counter for number of problems
\renewcommand{\homeworkProblemName}{#1} % Assign \homeworkProblemName the name of the problem
\section{\homeworkProblemName} % Make a section in the document with the custom problem count
\enterProblemHeader{\homeworkProblemName} % Header and footer within the environment
}{
\exitProblemHeader{\homeworkProblemName} % Header and footer after the environment
}

\newcommand{\problemAnswer}[1]{ % Defines the problem answer command with the content as the only argument
\noindent\framebox[\columnwidth][c]{\begin{minipage}{0.98\columnwidth}#1\end{minipage}} % Makes the box around the problem answer and puts the content inside
}

\newcommand{\homeworkSectionName}{}
\newenvironment{homeworkSection}[1]{ % New environment for sections within homework problems, takes 1 argument - the name of the section
\renewcommand{\homeworkSectionName}{#1} % Assign \homeworkSectionName to the name of the section from the environment argument
\subsection{\homeworkSectionName} % Make a subsection with the custom name of the subsection
\enterProblemHeader{\homeworkProblemName} % Header and footer within the environment
}{
\enterProblemHeader{\homeworkProblemName} % Header and footer after the environment
}

%----------------------------------------------------------------------------------------
%	NAME AND CLASS SECTION
%----------------------------------------------------------------------------------------

\newcommand{\hmwkTitle}{220CT Data and Information Retrieval\\ Coursework} % Assignment title
\newcommand{\hmwkDueDate}{March 13th 2015} % Due date
\newcommand{\hmwkClass}{ECU178 \- Computer Science} % Course/class
\newcommand{\hmwkClassTime}{10:30am} % Class/lecture time
\newcommand{\hmwkClassInstructor}{} % Teacher/lecturer
\newcommand{\hmwkAuthorName}{Robert Rigler} % Your name
\newcommand{\hmwkStudentID}{4939377}% My Student ID
\newcommand{\hmwkClassShort}{220CT Coursework } %Short name for class (only used in header)

%----------------------------------------------------------------------------------------
%	TITLE PAGE
%----------------------------------------------------------------------------------------

\title{
\vspace{2in}
\textmd{\textbf{\hmwkClass:\ \\ \hmwkTitle}}\\
\normalsize\vspace{0.1in}\small{Due\ on\ \hmwkDueDate}\\
\vspace{0.1in}\large{\textit{\hmwkClassInstructor\ }}
\vspace{3in}
}

\author{\textbf{\hmwkAuthorName\ : \hmwkStudentID}}
\date{} % Insert date here if you want it to appear below your name

%----------------------------------------------------------------------------------------

\begin{document}

\maketitle

%----------------------------------------------------------------------------------------
%	TABLE OF CONTENTS
%----------------------------------------------------------------------------------------

%\setcounter{tocdepth}{1} % Uncomment this line if you don't want subsections listed in the ToC

\newpage
\tableofcontents
\newpage

%----------------------------------------------------------------------------------------
%	Problem 1
%----------------------------------------------------------------------------------------
\begin{homeworkProblem}[Task 1 Database Design]
	This task in Database Design consists of four activities. The first three involve normalising the given data to third normal form, and the fourth is to produce and Entity Relationship diagram of the normalised relations.\\
		For each activity I will give a detailed step by step explanation of how I completed each activity. 
		
		
	\begin{homeworkSection}{Activity 1: First Normal Form}{
		
		To put this data into First Normal Form (1NF), I need to:
		\begin{enumerate}
			\item Identify any repeating redundant data and remove it from the current Entity
			\item Place the data into a new Entity
			\item Create a relationship with a primary key from one Entity as a foreign key in the other.
		\end{enumerate}
			
			\subsubsection{Step 1: Identify Redundancy}
				
		On inspecting the data, I can see that there are multiple instances of repeating data.\\ Orders' ID: \emph{CON-2237, CON-2356} and \emph{CON-1234}  all  have repeating data entries for fields:
		 \emph{Equipment, Qty,} and \emph{Unit Price} .
		 
		 \subsubsection{Step 2: Create New Entity}
		 
		 Removing the \emph{Equipment, Qty,} and \emph{Unit Price} fields and placing them in a new entity, leaves me with two entities as shown below.\\
		 
		
		 
		
		 
		 
		
		 
		\subsubsection{Step 3: Relationships and Keys}
		
		To complete the First Normal Form, a relationship needs to be created between the entities.
		
		 I created the relationship by including the \textit{Order ID} attribute as a foreign key in the \textit{ItemOrder} entity.\\
		\textit{Order ID} is used as a Primary Key for the \textit{Order} entity.
		In the \emph{ItemOrder} entity, no one attribute can be used to uniquely identify a single record. For this reason, I have created a concatenated key using the attributes \emph{Equipment} and \emph{Order ID}. The concatenated key can now be used to uniquely identify each record.\\
		
		\subsubsection{1NF : Diagram}
		 \begin{tabular}{l r| c c c c l r|} Order & $($ \underline{Order ID} & & & & & ItemOrder & $($ \underline{*Order ID}\\
		 	& Supplier ID	& & & & & & \underline{Equipment} \\
		 	& Client Name & & & & & & Qty $)$\\
		 	& Client Address  & & & & & & Unit Price $)$\\
		 	& Date\\
		 	& Total Price$)$\\
		 \end{tabular}
		 
		\pagebreak
		
		 \subsubsection{1NF :  Data}
		 \hspace{7.5cm}\textit{Order} entity
		 
		  \includegraphics[ width=1\columnwidth, height=3cm]{Task1/Order1NF.png}
		  
		  \vspace{10px}
		  
		 \hspace{4cm} \emph{ItemOrder} entity\\
		 \includegraphics[scale = 0.8]{Task1/Item11NF.png}
			}
		
		
	
	\end{homeworkSection}
	
	\pagebreak
	
	\begin{homeworkSection}{Activity 2: Second Normal Form}
		To put this data into the Second Normal Form(2NF) I need to ensure that the attributes are completely dependant on the primary key, i.e., that no attribute is only dependant on one part of the primary key.\\
		This can be done in two steps:
		\begin{enumerate}
			\item Test each attribute for complete dependency on the primary key.
			\item Remove any partially dependent attributes to a new entity and assign a primary key.
		\end{enumerate}
		
		For this particular set of data, the \emph{Order} entity does not have a concatenated key and therefore is already in the second normal form.
		
		\subsubsection{Step 1: Testing each attribute.} \vspace{10px}
		
		\begin{center}
			
			
			
			\begin{tabular}{|l|c|l|}
				\hline
				\textbf{	Primary Key} & \textbf{Attribute} & \textbf{Functionally Dependant?}\\ \hline
				Order ID, Equipment & Qty & Yes, dependant on both\\ \hline
				Order ID, Equipment & Unit Price & No, dependant on Equipment only\\ \hline
			\end{tabular}
			
		\end{center}
		\subsubsection{Step 2: Entity, Relationship and Key }
		
		From my testing, I found that the attribute \emph{Unit Price} is not functionally dependant as it is only dependent on \textit{Equipment}, but not\textit{ Order ID}.\\
		 I moved \emph{Equipment} and \textit{Unit Price} into a new entity called \textit{Item} and made \textit{Equipment} the primary key. \\ I created a relationship between the \textit{Item} and \textit{ItemOrder} entities by repeating the \textit{Equipment} attribute in \textit{ItemOrder} as a foreign key.
		
		\vspace{10px}
		
		
		
		\subsubsection{Entities in Second Normal Form: Diagram}
		
		\begin{tabular}{l r |c c c c l r| c c c l r|}
		
		{\textbf{Order}} &$($ \underline{Order ID} & & & & & \textbf{Item} & $($ \underline{Equipment} & & & & \textbf{ItemOrder}& $($\underline{*Order ID}  \\
			& Suppler ID & & & & & & Unit Price$)$ & & & & & \underline{*Equipment}\\
			& Client Name & & & & & & & & & & & Qty $)$\\
			& Client Address & & & & & \\
			& Date & & & & & \\
			& Total Price $)$ & & & & & \\
			  
		\end{tabular}
		\pagebreak
		\subsubsection{Entities in Second Normal Form: With data}
	\centering \textit{Order} entity
		
		\includegraphics[ width=1\columnwidth, height=3cm]{Task1/Order1NF.png}
		\vspace{20px}
		\begin{center}
				\textit{Item} entity
				
				\includegraphics[scale = 1]{Task1/Item02.png}
	
		
		\vspace{20px}
		\textit{ItemOrder} entity
		
		\includegraphics[scale = 0.7]{Task1/Item12NF.png}
		\end{center}
		
		\pagebreak
	\end{homeworkSection}
	\begin{homeworkSection}{Activity 3: Third Normal Form}
		
		To place the data into Third Normal Form 3NF I need to ensure that all attributes are only dependent on the Primary Key, and not Non-Key Attributes.
		This can be achieved in two steps:
		
		\begin{enumerate}
			\item Test each attribute for dependency on the primary key.
			\item Remove all transitive dependencies to a new entity with the correct primary key and relationship.
		\end{enumerate}
		
		For this particular set of data, both the \emph{Item} and \textit{ItemOrder} entities are already in the Third Normal Form.
		
		\subsubsection{Step 1: Testing for Transitive Dependency}
		
		\emph{Order} Entity
		
		\begin{tabular}{|l|l|l|}
			\hline
			\textbf{Primary Key} & \textbf{Attribute} & \textbf{Transitive Dependency?}  \\\hline
			Order ID & Supplier ID & Yes: Supplier ID can be found if we know Client Name or Address \\\hline
			Order ID & Client Name & Yes: Client Name can be found if we know Supplier ID or Address  \\\hline
			Order ID & Client Address & Yes: Client Address can be found if we know Client Name or Supplier ID   \\\hline
			Order ID & Date & No: Only dependent on Primary Key  \\\hline
			Order ID & Total Cost& No: Only dependent on Primary Key \\\hline
		\end{tabular}
		\vspace{10px}
		
		Using this table I have identified that \textit{Supplier ID, Client Name,} and\textit{ Client Address} all have transitive dependencies, and need to be moved to a new entity.
		
		\subsubsection{Step 2: Entity, Relationship and Key}
		
		I created a new entity called \textit{Customer}, and moved the three attributes with transitive dependencies into it. I then made \textit{Supplier ID} the Primary Key and created a relationship between the \textit{Customer} and \textit{Order} entities by repeating the \textit{Supplier ID} attribute as a foreign key in the \textit{Order} entity.\\
	
			\subsubsection{Entities in Third Normal Form: Diagram}
			
		\begin{tabular}{l r |c c c c l r| c c c l r|}
			
			{\textbf{Order}} &$($ \underline{Order ID} & & & & & \textbf{Item} & $($ \underline{Equipment} & & & & \textbf{ItemOrder}& $($\underline{*Order ID}  \\
			& *Suppler ID & & & & & & Unit Price$)$ & & & & & \underline{*Equipment}\\
			& Date& & & & & & & & & & & Qty $)$\\
			& Total Price $)$  
			
		\end{tabular}
		
		
				\vspace{40px}
				
		\begin{tabular}{l r|}
			\textbf{Customer} & $($ \underline{Supplier ID} \\
			& Client Name\\
			& Client Address $)$\\
		\end{tabular}		
		\pagebreak
		\subsubsection{Entities in Third Normal Form: With Data}
		\begin{center}
			
			\textit{Order} entity
			
			\includegraphics[scale = 0.8]{Task1/03.png}
			
			\vspace{10px}
			\textit{Item} entity
			
			\includegraphics[scale = 0.8]{Task1/I3.png}
			
			\vspace{20px}
			\textit{ItemOrder} entity
			
			
			\includegraphics[scale = 0.8]{Task1/IO3.png}
			
			\vspace{20px}
			\textit{Customer} entity
			
			\includegraphics[scale = 0.8]{Task1/C3.png}
		\end{center}
		\pagebreak
		
		
	\end{homeworkSection}
	\begin{homeworkSection}{Activity 4: ER Diagram}
		
		
	\end{homeworkSection}
	\end{homeworkProblem}

\begin{homeworkProblem}[Task 2: Database Development]
	For this task I will provide the SQL statement I used to complete each question activity, I will then explain each part of the SQL statement, and give screenshots of before and after the statement was executed (if applicable).\\
	Each SQL statement was writted and tested using ORACLE 11g Express Edition and ORACLE Application Express.
	\begin{homeworkSection}{1: Creating The Database Tables }
		
		To begin I created the two tables which have no dependency on any other table: \textit{Aircraft} and \textit{Airline}\\
		
		\subsubsection{Aircraft}
	\Cscript{Task2/C_a}{CREATE AIRCRAFT}
	
	The statement begins with \textit{'CREATE TABLE Aircraft'} which will create a table called \textit{aircraft}. Inside the brackets the three fields, their data types, and any constraints are listed. Below is a table which will explain why I chose these data types and constraints for each field.   
	
	\vspace{1.5cm}
	\begin{tabular}{|l|c|l|p{8cm}|}\hline \textbf{Identifier} & \textbf{Data-Type} &\textbf{Constraint} & \textbf{Explanation} \\ \hline 
		aircraft\textunderscore code & VARCHAR2(5) & PRIMARY KEY & All aircraft\textunderscore codes start with 'C' and are followed by up to four numerical digits. This field will be used as the Primary Key of the table.\\ 
		aircraft\textunderscore type & VARCHAR(30) & NOT NULL & Contains a combination of alphanumeric characters of up to 30 characters in length. This field cannot be left empty.\\
		aircraft\textunderscore price & NUMBER(11,2) & NOT NULL & Stores large numeric values, with a  precision of 11 and a scale of 2. This allows for prices up to 99 Thousand Million and two decimal places. This field cannot be left empty.\\\hline
		
	\end{tabular}
	\pagebreak
	
	
	\subsubsection{Airline}
	\Cscript{Task2/C_ac}{CREATE AIRLINE}
	
	\begin{tabular}{|l|l|l|p{8cm}|}\hline \textbf{Identifier} & \textbf{Data-Type} &\textbf{Constraint} & \textbf{Explanation} \\ \hline 
	airline\textunderscore code & CHAR(4) & PRIMARY KEY & A combination of 4 alphanumeric characters.    airline\textunderscore code is also the Primary Key\\
	airline\textunderscore name & VARCHAR2(20) & NOT NULL &  This field allows a variable length of characters, up to a maximum of 20. This field cannot be left empty. \\
	airline\textunderscore address & VARCHAR2(60) & NOT NULL & This field allows for a combination of alphanumeric characters up to a maximum length of 60. This field cannot be left empty.\\
	airline\textunderscore city & VARCHAR2(15) & NOT NULL & This field allows a combination of alphanumeric characters up to a maximum length of 15. This field cannot be left empty.\\
	airline\textunderscore country & VARCHAR2(15) & NOT NULL & This field allows for a combination of alphanumeric characters up to a maximum length of 15. This field cannot be left empty.\\
		\hline
	\end{tabular}
	
	\pagebreak
	
	\subsubsection{Purchase\textunderscore Order}
	

	\Cscript{Task2/c_P}{CREATE PURCHASE\textunderscore ORDER}
	
\begin{tabularx}{\textwidth}{|l|l|p{3cm}|X|}\hline \textbf{Identifier} & \textbf{Data-Type} &\textbf{Constraint} & \textbf{Explanation} \\ \hline 
	purchase\textunderscore order\textunderscore no & NUMBER(3,0) & PRIMARY KEY & This field allows for numeric input with a precision of 3 and a scale of 0, this allows values in the range of 001 to 999. This field also acts as the Primary Key.\\
	airline\textunderscore code & CHAR(4) & NOT NULL,\newline FOREIGN KEY & A combination of 4 alphanumeric characters. This field is a foreign key; Referencing the \textit{airline\textunderscore code} field from the \textit{Airline} table.\\
	date\textunderscore of\textunderscore purchase & DATE & NOT NULL\\
	\hline
\end{tabularx}
	
	
	\pagebreak
	\subsubsection{Ordered\textunderscore Aircraft}
	
	\Cscript{Task2/C_Oa}{CREATE ORDERED\textunderscore AIRCRAFT}
	
	
	\end{homeworkSection}
	
\end{homeworkProblem}
\end{document}