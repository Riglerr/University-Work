

%----------------------------------------------------------------------------------------
%	PACKAGES AND OTHER DOCUMENT CONFIGURATIONS
%----------------------------------------------------------------------------------------

\documentclass{article}

\usepackage{fancyhdr} % Required for custom headers
\usepackage{lastpage} % Required to determine the last page for the footer
\usepackage{extramarks} % Required for headers and footers
\usepackage[usenames,dvipsnames]{color} % Required for custom colors
\usepackage{graphicx} % Required to insert images
\usepackage{listings} % Required for insertion of code
\usepackage{courier} % Required for the courier font
\usepackage[table]{xcolor}
\usepackage{multirow}
\usepackage{tabularx}
\usepackage{hyperref}
\usepackage{lscape}

% Margins
\topmargin=-0.45in
\evensidemargin=0in
\oddsidemargin=0in
\textwidth=6.5in
\textheight=9.0in
\headsep=0.25in

\linespread{1.1} % Line spacing

% Set up the header and footer
\pagestyle{fancy}
\lhead{\hmwkAuthorName\ : \hmwkStudentID} % Top left header
\chead{ \hmwkClassShort} % Top center head
\rhead{\firstxmark } % Top right header
\lfoot{\lastxmark} % Bottom left footer
\cfoot{} % Bottom center footer
\rfoot{Page\ \thepage\ of\ \protect\pageref{LastPage}} % Bottom right footer
\renewcommand\headrulewidth{0.4pt} % Size of the header rule
\renewcommand\footrulewidth{0.4pt} % Size of the footer rule

\setlength\parindent{0pt} % Removes all indentation from paragraphs

%----------------------------------------------------------------------------------------
%	CODE INCLUSION CONFIGURATION
%----------------------------------------------------------------------------------------

\definecolor{MyDarkGreen}{rgb}{0.0,0.4,0.0} % This is the color used for comments
\lstloadlanguages{C} % Load Perl syntax for listings, for a list of other languages supported see: ftp://ftp.tex.ac.uk/tex-archive/macros/latex/contrib/listings/listings.pdf
\lstset{language=SQL, % Use c in this example
        frame=single, % Single frame around code
        basicstyle=\small\ttfamily, % Use small true type font
        keywordstyle=[1]\color{Blue}\bf, % cfunctions bold and blue
        keywordstyle=[2]\color{Purple}, % c function arguments purple
        keywordstyle=[3]\color{Blue}\underbar, % Custom functions underlined and blue
        identifierstyle=, % Nothing special about identifiers                                         
        commentstyle=\usefont{T1}{pcr}{m}{sl}\color{MyDarkGreen}\small, % Comments small dark green courier font
        stringstyle=\color{Purple}, % Strings are purple
        showstringspaces=false, % Don't put marks in string spaces
        tabsize=5, % 5 spaces per tab
        %
        % Put standard c functions not included in the default language here
        morekeywords={rand},
        %
        % Put c function parameters here
        morekeywords=[2]{on, off, interp},
        %
        % Put user defined functions here
        morekeywords=[3]{test},
       	%
        morecomment=[l][\color{Blue}]{...}, % Line continuation (...) like blue comment
        numbers=left, % Line numbers on left
        firstnumber=1, % Line numbers start with line 1
        numberstyle=\tiny\color{Blue}, % Line numbers are blue and small
        stepnumber=1 % Line numbers go in steps of 5
}

% Creates a new command to include a script, the first parameter is the filename of the script (without .txt), the second parameter is the caption
\newcommand{\Cscript}[2]{
\begin{itemize}
\item[]\lstinputlisting[caption=#2,label=#1]{#1.txt}
\end{itemize}
}
\renewcommand{\arraystretch}{1.5}
\newcommand{\tab}[1]{\hspace{.08\textwidth}{#1}}
%----------------------------------------------------------------------------------------
%	DOCUMENT STRUCTURE COMMANDS
%	Skip this unless you know what you're doing
%----------------------------------------------------------------------------------------

% Header and footer for when a page split occurs within a problem environment
\newcommand{\enterProblemHeader}[1]{
\nobreak\extramarks{#1}{#1 continued on next page\ldots}\nobreak
\nobreak\extramarks{#1 }{#1 continued on next page\ldots}\nobreak
}

% Header and footer for when a page split occurs between problem environments
\newcommand{\exitProblemHeader}[1]{
\nobreak\extramarks{#1}{#1 continued on next page\ldots}\nobreak
\nobreak\extramarks{#1}{}\nobreak
}

\setcounter{secnumdepth}{0} % Removes default section numbers
\newcounter{homeworkProblemCounter} % Creates a counter to keep track of the number of problems

\newcommand{\homeworkProblemName}{}
\newenvironment{homeworkProblem}[1][Problem \arabic{homeworkProblemCounter}]{ % Makes a new environment called homeworkProblem which takes 1 argument (custom name) but the default is "Problem #"
\stepcounter{homeworkProblemCounter} % Increase counter for number of problems
\renewcommand{\homeworkProblemName}{#1} % Assign \homeworkProblemName the name of the problem
\section{\homeworkProblemName} % Make a section in the document with the custom problem count
\enterProblemHeader{\homeworkProblemName} % Header and footer within the environment
}{
\exitProblemHeader{\homeworkProblemName} % Header and footer after the environment
}

\newcommand{\problemAnswer}[1]{ % Defines the problem answer command with the content as the only argument
\noindent\framebox[\columnwidth][c]{\begin{minipage}{0.98\columnwidth}#1\end{minipage}} % Makes the box around the problem answer and puts the content inside
}

\newcommand{\homeworkSectionName}{}
\newenvironment{homeworkSection}[1]{ % New environment for sections within homework problems, takes 1 argument - the name of the section
\renewcommand{\homeworkSectionName}{#1} % Assign \homeworkSectionName to the name of the section from the environment argument
\subsection{\homeworkSectionName} % Make a subsection with the custom name of the subsection
\enterProblemHeader{\homeworkProblemName} % Header and footer within the environment
}{
\enterProblemHeader{\homeworkProblemName} % Header and footer after the environment
}

%----------------------------------------------------------------------------------------
%	NAME AND CLASS SECTION
%----------------------------------------------------------------------------------------

\newcommand{\hmwkTitle}{220CT Data and Information Retrieval\\ Case Study\\ Time to dig deeper!} % Assignment title
\newcommand{\hmwkDueDate}{March 20th 2015} % Due date
\newcommand{\hmwkClass}{ECU178 \- Computer Science} % Course/class
\newcommand{\hmwkClassTime}{10:30am} % Class/lecture time
\newcommand{\hmwkClassInstructor}{} % Teacher/lecturer
\newcommand{\hmwkAuthorName}{Robert Rigler} % Your name
\newcommand{\hmwkStudentID}{4939377}% My Student ID
\newcommand{\hmwkClassShort}{220CT Time to dig deeper! } %Short name for class (only used in header)

%----------------------------------------------------------------------------------------
%	TITLE PAGE
%----------------------------------------------------------------------------------------

\title{
\vspace{2in}
\textmd{\textbf{\hmwkClass:\ \\ \hmwkTitle}}\\
\normalsize\vspace{0.1in}\small{Due\ on\ \hmwkDueDate}\\
\vspace{0.1in}\large{\textit{\hmwkClassInstructor\ }}
\vspace{3in}
}

\author{\textbf{\hmwkAuthorName\ : \hmwkStudentID}}
\date{} % Insert date here if you want it to appear below your name

%----------------------------------------------------------------------------------------

\begin{document}

\maketitle

%----------------------------------------------------------------------------------------
%	TABLE OF CONTENTS
%----------------------------------------------------------------------------------------

%\setcounter{tocdepth}{1} % Uncomment this line if you don't want subsections listed in the ToC

\newpage
\tableofcontents
\newpage

%----------------------------------------------------------------------------------------
%	Problem 1
%----------------------------------------------------------------------------------------
\begin{homeworkProblem}[Data Mining Definitions]
	\begin{homeworkSection}{Forecasting}
	\subsubsection{Definition: }
\tab	Forecasting is the act of predicting a future activity by the analysis of past and present data trends. This is done via the visualisation of relevant data-set, so to derive patterns and 
	\end{homeworkSection}
	
	\begin{homeworkSection}{Regression}
		\subsubsection{Definition: }
	\tab	Regression is the application of certain equations to a data-set in order to better model the data.\\
	
	For example applying the equation of a straight line ( $y=mx+c$) to a data-set and seeing if it forms a correlation to a straight line, then future values of \textit{y} and \textit{x} can be predicted.
	\end{homeworkSection}
	
	\begin{homeworkSection}{Time Series}
		
		\subsubsection{Definition: }
	\tab	A time series is a collection of data that has been collected at regular, sequential intervals over a specific period of time. Time-Series analysis is the use of statistical techniques to model and derive patterns from that data set.
		
		Time-series can be applied in many different fields, but most notably:
		\begin{itemize}
			\item Financial:\\
				\tab  Forecasting inflation and stock prices.
			\item Scientific:\\
				Forecasting results from experiments.
		\end{itemize}
	\end{homeworkSection}
	
	\begin{homeworkSection}{Associtaiton}
	\subsubsection{Definition: }
	\tab Defining a set of rules, that imply a relationship(an association) between data items.
	
	By analysis of a dataset, certain relationships will be discovered. Using these relations, rules can be made which can help the forecasting of future data.
	\end{homeworkSection}
	
	\begin{homeworkSection}{Sequencing}
		\subsubsection{Definition: }
	\tab The act of analysing a dataset for sequences of action, and then from this data find the set of most frequent sequences.
	\end{homeworkSection}
\end{homeworkProblem}
\pagebreak
\begin{homeworkProblem}[Data Mining and Music]
	Data-Mining and music are not regularly linked together when studying either of the topics. But with digital music distribution (via download and streaming) being the most lucrative method of distribution.\\
	
	Digital Music is a very popular business,  a lot of money is made via advertising and downloads. To keep customers' loyalty , a recommendation system is often used to notify the users of other music that they might be interested in. The more music that the user is aware of, the greater potential profit increases. 
	Currently, data-mining in music is most used in music streaming, where users can create their own 'Radio Station'. The user chooses an artist or a genre and the service will then play songs that are similar to the song or match the given criteria, the user can then 'like' or 'dislike' the current song, and the service will improve the station based on that information.
	
	\begin{homeworkSection}{How?}
		Firstly, the data is gathered. This comes in many formats:\\
		Meta data: Name, Genre, Label, Sales.\\
		Associations: The various relationships between artists, genres and labels.\\
		User Data: Although this is less relevant, because individual tastes can not be necessarily applied to others especially in music.\\
		
		After all the relevant data has been gathered the data is then grouped into similarities and then  analysed.\\
		The data can be made into a graph and then similarity relationships can be put into place. The relationships point to other nodes on the graph (called neighbours). The graph can then be traversed so similar music can be found.\\
		
		The individual data items could also be represented as a matrix which could be populated by how many times two song were recommended, or played sequentially or appeared together. From the matrix a distance function (Regression) could be used to analyse how similar those two items are.\\
		A similar approach would be to model the user as a vector. The music played by this user influence the direction and magnitude of this vector, which can then be used to predict similarities.
		
	\end{homeworkSection}
	
	
\end{homeworkProblem}
\end{document}